%%%%%%%%%%%%%%%%%%%%%%%%%%%%%%%%%%%%%%%%%
% Press Release
% LaTeX Template
% Version 1.0 (2/6/13)
%
% This template has been downloaded from:
% http://www.LaTeXTemplates.com
%
% Original author:
% Vel (vel@latextemplates.com)
%
% License:
% CC BY-NC-SA 3.0 (http://creativecommons.org/licenses/by-nc-sa/3.0/)
%
%%%%%%%%%%%%%%%%%%%%%%%%%%%%%%%%%%%%%%%%%

%----------------------------------------------------------------------------------------
%	PACKAGES AND OTHER DOCUMENT CONFIGURATIONS
%----------------------------------------------------------------------------------------

\documentclass[11pt,pressrelease]{newlfm} % Font size

\usepackage{charter} % Use the Charter font for the document text

% \PhrPhone{Phone} % Customize the "Telephone" text
\PhrEmail{Email} % Customize the "E-mail" text
%\PhrContact{Contact} % Uncomment this line to change the 'Contact:' text

%----------------------------------------------------------------------------------------
%	PRESS RELEASE INFORMATION
%----------------------------------------------------------------------------------------

% \makeletterhead{Uiuc}{\Cheader{\vspace{16pt}\includegraphics[width=0.5\linewidth]{logo.png}}} % Include a company logo, if you don't use one you will need to uncomment line 6 in the prsrls.tex file
% \lthUiuc % Print the company/institution logo

\release{Cassowary PureScript Proposal} % When the press release may be used

\namefrom{Athan Clark} % Name

% \addrfrom{ % From address
% 123 Broadway \\
% City, State 12345
% }

% \phonefrom{(000) 111-1111} % Phone number
%
\emailfrom{athan.clark@gmail.com} % Email address

\headline{Pure Functional Programming for Total Correctness} % Headline for the press release

\newcommand{\subtitle}{The benefits of functional languages like PureScript for implementing complex,
mathematically dense concepts, like constraint solvers or compilers.} % Subtitle for the press release, if you don't want one just remove the subtitle text leaving the rest of the command

% \byline{\textbf{City, Country -- \today ~--} Summary/overview of facts including who, what, where, when as well as why the press should care enough to write about this.} % A summary line for the press release

%----------------------------------------------------------------------------------------

\begin{document}
\begin{newlfm}

%----------------------------------------------------------------------------------------
%	PRESS RELEASE CONTENT
%----------------------------------------------------------------------------------------

\begin{singlespace} % Uncomment for single line spacing

It has been recently discovered that computer science is the heart of mathematics and logic.
However, this is true only for functional programming languages like Haskell and PureScript -
they do away with the concept of a machine, and direct all attention to the concept of a
mathematical function. Programming then becomes the creation of logical specifications and
stateless functions, rather than sequentially manipulating data.

\section{Correctness}

Through functional programming, you can be garunteed software correctness. The insurance comes from the
precise specifications of \textit{types}, detailing the behavior of stateless functions. If an
inconsistency is found, the program will simply not compile - it is a safeguard far
more effective than unit tests. By establishing \textit{algebraic properties} (such as associativity and
commutativivty of expressions), we can \textbf{eliminate large classes of bugs}
from being possible.

\section{Performance}

Functional languages like Haskell or PureScript find an elegant balance between formal correctness
and performance. By relying on purity - the lack of affecting reality - one is free to
\textbf{optimize at a macroscopic level}. This is evident in the premier Haskell compiler, GHC,
where you can get better SIMD performance than hand and machine-optimized C.
Likewise, Warp (a web server written in Haskell) can achieve over
20,000 responses per second on a \$200 laptop, without optimizations enabled, compared to
Node.js getting only 600 rps on the same machine.

\section{Clarity}

In addition to more correct software and greater performance, purely functional programs
are smaller and more concise than their impertative counterparts. With less room for human error
and more ability to express ideas, writing many useful algorithms like quicksort and
factorial can be done in a single line of code.

\section{Examples}

The benefits of using such a language are obvious for any software project, \textit{especially}
when problems get complex.

CompCert is a C compiler, similar to GCC, implemented in a functional programming language.
It's performance is not as good as GCC's, but it is almost there - about 80\%.
But the size of the source code is remarkable:
CompCert is at about 60,000 lines of code - 50,000 dedicated to exhaustive proofs (verifying
correctness of binary translations on all supported architectures), with about 8000 lines
actually implementing the compiler. This is hardly a large codebase, especially compared
to the 15 million lines of code behind GCC.

Less code \textit{may} mean less room for bugs, but the real reason for functional programming's
elegance is its underlying theory - one can directly translate category theory specifications to programs, or
formal proofs in propositional logic to types, in an identical syntax. Cryptol is a great
example of how someone can define cryptographic algorithms that mirror their algebraic specficiations,
one-for-one.

\section{Conclusion}

In essence, if you are doing anything complicated, it is best to have a mathematical concept of
your problem. And if you have a mathematical definition of a problem, it is \textbf{trivial} to
implement the algorithm in a functional language, because it is so closely related to
mathematics itself. There have been many successful projects implemented in pure functional
programming languages - automated trading systems for the stock market, compilers and language
interpreters, constraint solvers, and a wide array of mathematically dense algorithms.

Cassowary is a great example of an algorithm that has rich mathematical heritage, but has
grown rotten from neglect - it is simply too difficult to take a high-level algebraic concept
and implement it in an imperative setting. But with the rise of pure functional programming,
I am confident that I can make a smaller, more correct, and more performant implementation of the
Cassowary algorithm in PureScript for client-side browsers.

\end{singlespace} % Uncomment for single line spacing

%----------------------------------------------------------------------------------------

\end{newlfm}
\end{document}
