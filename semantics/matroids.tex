\documentclass{article}
\usepackage{amsmath}

\begin{document}


\title{Intuitionistic Approach to Matroids}
\author{Athan Clark\\ Copyright \copyright \enspace The Grid, 2015}

\maketitle

\begin{abstract}
Matroids are an important interest to (combinatorial) optimization - where each
"step'' made strictly approaches the optimal solution. Here we assert the properties
and definition of matroids in an intuitionistic setting, to better formalize and
prove matroids as correct and important.
\end{abstract}

\section{Overview}

As one definition, a Matroid \(M\) consists of a \textbf{ground} set \(E\) and a "family''
\(I\) of \textbf{independent} subsets of \(E\), with the following properties:

\[
             E : \forall \sigma. \enspace Set \enspace \sigma
  , \enspace I : \forall \sigma. \enspace Set \enspace (Set \enspace \sigma) \quad \mathrm{where}
\]

\begin{flalign}
  &I \not \equiv \emptyset \iff \emptyset \in I \label{matroid-nonempty} \tag{MATROID-NONEMPTY}\\
  &\forall a \in A. \enspace a \subseteq E \label{matroid-subset} \tag{MATROID-SUBSET}\\
  &\quad \forall \alpha \subseteq a. \enspace \alpha \in E \label{matroid-hereditary} \tag{MATROID-HEREDITARY}\\
  &\forall i_{n}, \enspace i_{n+1} \in I. \enspace \exists ! e \in i_{n+1} - i_{n} \nonumber\\
  &\quad \forall i \in I. \enspace i \cup \{e\} \in I \label{matroid-growth} \tag{MATROID-GROWTH}
\end{flalign}

Where in \ref{matroid-growth}, \(i_{n}\) denotes the element \(i\) of \(I\) such that
its size \(|i_{n}| = n\); if there is \textit{one} element difference between the two
sets, then the differing element exists in all independent sets.


\end{document}
