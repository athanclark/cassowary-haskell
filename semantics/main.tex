\documentclass{article}
\usepackage{amsmath}

\begin{document}


\title{Properties of Cassowary}
\author{Athan Clark}

\maketitle

\begin{abstract}
This paper details the entailed logical properties that a linear constraint solver,
annotated with weights and error variables, should have. We hope to use this document
as a paper implementation of the automated test suites accompanied with the
Haskell and PureScript versions of Cassowary.
\end{abstract}


% Equations
\section{Equations}

The generic form for a linear inequality consists of a unique set of variables
summed together, and a constant value:

\begin{flalign}
  &\mathrm{newtype} \enspace LinVarMap \enspace \alpha \enspace = \enspace
                             LinVarMap \enspace (Map \enspace LinVarName \enspace \alpha) \label{linvarmap-def} \tag{LINVARMAP-DEF}\\
  &\mathrm{data} \enspace IneqExpr \enspace \alpha \enspace = \enspace IneqExpr \nonumber \\
  &\quad \{ \enspace coeffs \enspace :: \enspace (LinVarMap \enspace \alpha) \nonumber \\
  &\quad  , \enspace const \enspace :: \enspace Rational \enspace \} \label{ineqexpr-def} \tag{INEQEXPR-DEF}
\end{flalign}

We segregate the different forms of inequality expressions with newtypes:

\begin{flalign}
  &\mathrm{newtype} \enspace Equ \enspace \alpha \enspace = \enspace Equ \enspace (IneqExpr \enspace \alpha) \label{equ-def} \tag{EQU-DEF}\\
  &\mathrm{newtype} \enspace Lte \enspace \alpha \enspace = \enspace Lte \enspace (IneqExpr \enspace \alpha) \label{lte-def} \tag{LTE-DEF}\\
  &\mathrm{newtype} \enspace Gte \enspace \alpha \enspace = \enspace Gte \enspace (IneqExpr \enspace \alpha) \label{gte-def} \tag{GTE-DEF}
\end{flalign}

Note that the equations are polymorphic in their coefficient type - this will be
important when we introduce weights in section \ref{section-weights}.


% Simplex
\section{Simplex}
There are several properties for dual and primal simplex method and Bland's ratio.

\begin{equation}
    \label{simple_equation}
    \alpha = \sqrt{ \beta }
\end{equation}

\subsection{Subsection Heading Here}
Write your subsection text here.


% Weights
\section{Weights} \label{section-weights}

Weights are implemented as a (non-empty) list of coefficients:

\begin{flalign}
  &\mathrm{newtype} \enspace Weight \enspace \alpha \enspace = \enspace [\alpha] \label{weight-def} \tag{WEIGHT-DEF}
\end{flalign}

When an equation using \(Rational\) values as coefficents gets augmented with a
weight (usually with a natural number - \(augment \enspace eq1 \enspace 5\)), the coefficients are
pushed to that index in an empty stream of \(0\)s; the example just mentioned
would stream five \(0\)s before containing the original coefficient.

\subsection{Arithmetic}

\subsubsection{Addition}

\textbf{Instances}:
\begin{flalign}
  &(.+.) \enspace :: \enspace Weight \enspace Rational \enspace \rightarrow
                     \enspace Weight \enspace Rational \enspace \rightarrow
                     \enspace Weight \enspace Rational \label{add-sym} \tag{ADD-SYM}
\end{flalign}

Addition in \ref{add-sym} is implemented with \(unionWith\) - leaving the larger
of the two lists intact.

\[
  (.+.) \enspace = \enspace \mathrm{unionWith} \enspace (+)
\]

\textbf{Lemma}: The length of the resulting list, when using addition, is the
                maximum length of the two lists added.

\[
  \mathrm{length} \enspace ( xs \enspace .+. \enspace ys ) \equiv
  \mathrm{max} \enspace ( length \enspace xs ) \enspace ( length \enspace ys )
\]

\subsubsection{Subtraction}

\textbf{Instances}:
\begin{flalign}
  &(.-.) \enspace :: \enspace Weight \enspace Rational \enspace \rightarrow
                     \enspace Weight \enspace Rational \enspace \rightarrow
                     \enspace Weight \enspace Rational \label{sub-sym} \tag{SUB-SYM}\\
  &(.-.) \enspace :: \enspace Rational \enspace \rightarrow
                     \enspace Weight \enspace Rational \enspace \rightarrow
                     \enspace Rational \label{sub-forget:1} \tag{SUB-FORGET-1}\\
  &(.-.) \enspace :: \enspace Weight \enspace Rational \enspace \rightarrow
                     \enspace Rational \enspace \rightarrow
                     \enspace Rational \label{sub-forget:2} \tag{SUB-FORGET-2}
\end{flalign}

For the first instance \ref{sub-sym}, we use \(unionWith\) again:

\[
  (.-.) \enspace = \enspace unionWith \enspace (-)
\]

For \ref{sub-forget:1} and \ref{sub-forget:2}, we sum the list (and forget weight data) before subtracting:

\begin{flalign*}
  &x \enspace .-. \enspace ys \enspace = \enspace x \enspace - \enspace sum \enspace ys\\
  &xs \enspace .-. \enspace y \enspace = \enspace sum \enspace xs \enspace - \enspace y
\end{flalign*}

\subsubsection{Multiplication}

\textbf{Instances}:
\begin{flalign}
  &(.*.) \enspace :: \enspace Weight \enspace Rational \enspace \rightarrow
                     \enspace Rational \enspace \rightarrow
                     \enspace Weight \enspace Rational \label{mul-dist:1} \tag{MUL-DIST-1}\\
  &(.*.) \enspace :: \enspace Rational \enspace \rightarrow
                     \enspace Weight \enspace Rational \enspace \rightarrow
                     \enspace Weight \enspace Rational \label{mul-dist:2} \tag{MUL-DIST-2}\\
  &(.*.) \enspace :: \enspace Weight \enspace Rational \enspace \rightarrow
                     \enspace Weight \enspace Rational \enspace \rightarrow
                     \enspace Weight \enspace Rational \label{mul-forget} \tag{MUL-FORGET}
\end{flalign}

\ref{mul-dist:1} and \ref{mul-dist:2} naturally distributes the \(Rational\)
multiplied value to every element in the \(Weight\) list. In the \ref{mul-forget}
instance, one of the arguments must be forgotten, and is therefore ambiguous for
it's behaviour. We leave the implementation for this instance ambiguous, only
necessary for implementing substitution.

\begin{flalign*}
  &xs \enspace .*. \enspace y \enspace = \enspace (* \enspace y) \enspace <\$> \enspace xs\\
  &x \enspace .*. \enspace ys \enspace = \enspace (x \enspace *) \enspace <\$> \enspace ys
\end{flalign*}

\subsubsection{Division}

\textbf{Instances}:
\begin{flalign}
  &(./.) \enspace :: \enspace Rational \enspace \rightarrow
                     \enspace Weight \enspace Rational \enspace \rightarrow
                     \enspace Rational \label{div-forget} \tag{DIV-FORGET}
\end{flalign}

We will need to divide a coefficient (\(Rational\)) by a coefficient (possibly
a \(Weight Rational\)) in \(blandRatioPrimal\), where we have a forgetful instance -
divide the constant by the sum of the error coefficients:

\[
  x \enspace ./. \enspace ys \enspace = \enspace x \enspace / \enspace sum \enspace ys
\]


\section{Conclusion}
Write your conclusion here.

\end{document}
