\documentclass{article}
\usepackage{amsmath}

\begin{document}


\title{Bland's Rule Elaboration}
\author{Athan Clark\\ Copyright \copyright \enspace The Grid, 2015}

\maketitle

\begin{abstract}
Bland's rule is the most popular method to deciding the next pivot to take when
refactoring a system of linear inequalities for more-optimized basic-feasible
solutions. Here, we detail Bland's rule, and provide it's dual form for implementation
in the Haskell version of the Cassowary constraint solver.
\end{abstract}

\section{Overview}

Say we have an objective function in the following form:

\begin{flalign*}
  &\omega \enspace = \enspace \sum_{0}^{n} \alpha_{n} x_{n} \enspace + \enspace c_{\omega}
\end{flalign*}

That is, an equation where the unique objective variable \(\omega\) is in basic-normal form, and
the rest of the public \(n\) variables are summed with their coefficients - \(\alpha_{n} x_{n}\),
and the constant value \(c_{\omega}\). We use \(\alpha\) to denote \textit{objective} coefficients.

This objective function is then maximized over a constraint set with the following form:

\begin{flalign*}
  & \left \{
      \sum_{0}^{n} \beta_{(m,n)} x_{n} \enspace + \enspace c_{m}
    \right \}
\end{flalign*}

For each constraint \(m\). We use \(\beta\) to denote \textit{constraint} coefficients. Each
constraint \(m\) can have any variable \(x_{n}\) in basic-feasible form. Also, note that
the slack variables normally introduced in this context are implicit - they are referenced
by the unique identifier \(m\) per-constraint.

\section{Bland's Rule Primal}

The primal version of Bland's rule proceeds as follows: Choose a column \(k\) out of
\(n\), and a row \(j\) out of \(m\) as the next variable to refactor into basic-normal
form over the whole constraint set - this as, \(x_{k}\) will be defined in terms of
the constraint \(j\).

\subsection{Column Selection}

\begin{flalign*}
  & \mathrm{Choose} \enspace \alpha_{k} \enspace := \enspace
    \mathrm{min} \left \{
                    \alpha_{n}
                    \enspace | \enspace \alpha_{n} < 0
                 \right \}
\end{flalign*}

We choose the column \(k\) to be the minimum \textit{negative} objective coefficient.

\subsection{Row Selection}

In the primal version of Bland's rule, we need to first solve for \(k\) before we
can solve for \(j\).

\begin{flalign*}
  & \mathrm{Choose} \enspace \frac{c_{j}}{\beta_{(j,k)}} \enspace := \enspace
    \mathrm{min} \left \{
                    \frac{c_{m}}{\beta_{(m,k)}}
                    \enspace \bigg| \enspace \beta_{(m,k)} > 0
                 \right \}
\end{flalign*}

That is, we choose \(j\) to be the minimum (primal) Bland ratio - the constant
\(c_{m}\) divided by the \textit{strictly positive} coefficient \(\beta_{(m,k)}\).

You can find the proofs that this method of row and column selection does not
cycle, and is strictly maximizing the basic-feasible solution in the literature.


\section{Bland's Rule Dual}

The dual version of Bland's rule is very similar to the primal version, except
for the swapped roles - this would behave the same as the primal version if we
transpose the matrix of the objective function and constraint coefficients.

Here, we will instead need to select the row \(j\) \textbf{first}, and the column
second.

\subsection{Row Selection}

\begin{flalign*}
  & \mathrm{Choose} \enspace c_{j} \enspace := \enspace
    \mathrm{min} \left \{
                    c_{m}
                    \enspace | \enspace c_{m} < 0
                 \right \}
\end{flalign*}

We choose the row \(j\) to be the minimum \textit{negative} constraint constant.

\subsection{Column Selection}

In the dual version of Bland's rule, we need to first solve for \(j\) before we
can solve for \(k\).

\begin{flalign*}
  & \mathrm{Choose} \enspace \frac{\alpha_{k}}{\beta_{(j,k)}} \enspace := \enspace
    \mathrm{min} \left \{
                    \frac{\alpha_{n}}{\beta_{(j,n)}}
                    \enspace \bigg| \enspace \beta_{(j,n)} > 0
                 \right \}
\end{flalign*}

The dual version of Bland's rule will \textit{minimize} the basic-feasible solution
for the constraint set and objective function.

\end{document}
