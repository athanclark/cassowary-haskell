\documentclass{article}
\usepackage{amsmath}
\usepackage{amssymb}
\usepackage{amsthm}

\renewcommand\qedsymbol{$\blacksquare$}
\newtheorem{theorem}{Theorem}

\newcommand{\where}{\enspace \mathrm{where} \enspace}
\newcommand{\hdo}{\enspace \mathrm{do} \enspace}
\newcommand{\hreturn}{\enspace \mathrm{return} \enspace}
\newcommand{\hif}{\enspace \mathrm{if} \enspace}
\newcommand{\hthen}{\enspace \mathrm{then} \enspace}
\newcommand{\helse}{\enspace \mathrm{else} \enspace}


\begin{document}


\title{Cassowary Development Report}
\author{Athan Clark}

\maketitle

\begin{abstract}
  The development of a pure implementation of the Cassowary constraint solver
  has been a difficult one indeed; I've found there to be crucial details
  overlooked in the specification of the algorithm - mainly regarding the
  concept of \textit{weighted} coefficients.

  I've further refined the concept of the algorithm down to a cleaner specification,
  with a few issues still remaining before a fast and efficient implementation
  can be achieved.
\end{abstract}

\section*{Main Issues}

\subsection*{Weighted Coefficients as Lists}

Initally, I was under the impression that weighted coefficients could be
tangibly regarded as "objects" - things that we can compare to each other,
combine, etc. This is actually not the case for Cassowary; rather, we should
imagine each weighted coefficient as a stream of values. Now when we need
to find the next pivot, instead of having a one-shot computation to find
the output we're looking for, we need to fold over all the streams at once.
This took me a while to understand, and all my code is based on my previous idea.

The Java implementation of Cassowary actually goes against this paradigm, and
naively tries the monolithic appoach I initially imagined - they crudely
multiply each weight by some large factor like 1000 I think it was. This is
incorrect according to the specification, and shouldn't be called a \textit{weighted}
Cassowary implementation.

The real perspective we should have is that the weights add a dimension to the
entire solving process, and we can't factor the entirity of that dimension in
each pivot step, but rather we would need to pivot multiple times along the
length of that dimension.

Luckily this won't be too difficult to engineer around - most of my code
is very generic and reusable, the main hurdle I see is retaining the generic
nature of the existing code while embedding this particular behavior. I still
need to work this out.

\subsection*{Oriented Matroid}

The oriented matroid embedding is still very elusive - I did a lot of work
making generic set-oriented optimization utilities for Haskell, and they
are very straightforward for abstract code. However, I still don't completely
understand how Richard Bland proved that his pivoting algorithm (the one that's
used in Cassowary) satisfies an oriented matroid.

Without this information, it would be hard to claim that my implementation would
be perfectly "correct". I can infer where I \textit{ought} to be coding, and
how this all ought to work, but I can't formally verify that my implementation
is an oriented matroid without \textbf{excessive} work.

I think I can get by without needing this, however in the long run it would
be wise to prove these formalisms for the library's integrity and stability.
For now thought, this is secondary to getting the system running.

\section*{Conclusion}

After I have the top point complete, then we will have (\textit{fast}) primal and dual
simplex optimization algorithms for \textit{weighted} constraint sets, which is 90\%
of what the Cassowary solver actually is under-the-hood.

After this, the rest of the engine's development will involve edge-case scenarios -
short circuiting unnecessary computations, small optimization tricks for edit constraints,
etc. Even this isn't that important - we could start working on the Haskell $\rightarrow$
PureScript port immediately if the Haskell version proves to be performant enough.

From my most recent benchmarks, we are working in the microsecond ($\mu s$) range
for single pivots on constraint sets \textit{without} weights, up to 1000 variables.
I haven't viewed the asymptotics of multiple pivots, nor how weights affect the scheme,
but I have high hopes.




\end{document}
